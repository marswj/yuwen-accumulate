\chapter{一年级(上)}
\section{古诗}
\subsection{咏鹅}

\begin{pinyinscope}
\noindent \huge 咏鹅

\noindent {\kaishu \large [唐]骆宾王} \\

\noindent 鹅鹅鹅,\\
曲项向天歌。\\
白毛浮绿水,\\
红掌拨清波。
\end{pinyinscope}

译文:
白天鹅啊白天鹅,脖颈弯弯,向天欢叫,洁白的羽毛,漂浮在碧绿水面;红红的脚掌,拨动着清清水波。

\subsection{江南}

\begin{pinyinscope}
\noindent \huge 江南

\noindent {\kaishu \large [汉]汉\xpinyin{乐}{yue4}府} \\

\noindent 江南可采莲,\\
莲叶何田田。\\
鱼戏莲叶间。\\
鱼戏莲叶东,\\
鱼戏莲叶西,\\
鱼戏莲叶南,\\
鱼戏莲叶北。
\end{pinyinscope}

译文:
江南水上可以采莲,莲叶多么茂盛,鱼儿在莲叶间嬉戏。
鱼在莲叶的东边游戏,鱼在莲叶的西边游戏,鱼在莲叶的的南边游戏,鱼在莲叶的北边游戏。


\subsection{画}
\begin{pinyinscope}
	\noindent \huge 画
	
	\noindent {\kaishu \large [唐]王维 } \\
	
	\noindent 远看山有色,\\
	近听水无声。\\
	春去花还在,\\
	人来鸟不惊。
\end{pinyinscope}

译文:
远看高山色彩明亮,走近一听水却没有声音。
春天过去,可是依旧有许多花草争奇斗艳,人走近,可是鸟却依然没有被惊动。

\subsection{悯农(其二)}
\begin{pinyinscope}
	\noindent \huge 悯农 {\kaishu \large (其二) }
	
	\noindent {\kaishu \large [唐]李绅 } \\
	
	\noindent 锄禾日当午,\\汗滴禾下土。\\
	\xpinyin{谁}{shui2}知盘中餐,\\粒粒皆辛苦。
\end{pinyinscope}

译文:
农民在正午烈日的暴晒下锄禾,汗水从身上滴在禾苗生长的土地上。
又有谁知道盘中的饭食,每颗每粒都是农民用辛勤的劳动换来的呢?

\subsection{古朗月行}
\begin{pinyinscope}
	\noindent \huge 古朗月行 {\kaishu \large (节选) }
	
	\noindent {\kaishu \large [唐]李白} \\
	
	\noindent 小时不\xpinyin{识}{shi2}月,\\呼作白玉盘。\\
	又疑瑶台镜,\\飞在青云端。
\end{pinyinscope}

译文:
小时不识天上明月,把它称为白玉圆盘。怀疑它是瑶台仙镜,飞在夜空青云上边。

\subsection{风}
\begin{pinyinscope}
	\noindent \huge 风
	
	\noindent {\kaishu \large [唐]李\xpinyin{峤}{qiao2} } \\
	
	\noindent 解落三秋叶,\\能开二月花。\\
	过江千尺浪,\\入竹万竿斜。
\end{pinyinscope}

译文:
能吹落秋天金黄的树叶,能吹开春天美丽的鲜花。
刮过江面能掀千尺巨浪,吹进竹林能使万竿倾斜。

\section{日积月累}
\subsection{积累1}
\begin{pinyinscope}
\huge
\noindent \xpinyin{一}{yi4}年之计在于春,\\
\xpinyin{一}{yi2}日之计在于晨。\footnote{南朝梁·萧统《纂要》}​\\

\noindent \xpinyin{一}{yi2}寸光阴\xpinyin{一}{yi2}寸金,\\
寸金难买寸光阴。
\end{pinyinscope}

\subsection{积累2}
\begin{pinyinscope}
	\huge
	\noindent \xpinyin{种}{zhong4}瓜得瓜,\xpinyin{种}{zhong4}豆得豆。\\
	
	\noindent 前人栽树,后人乘凉。\\
	
\noindent	千里之行,始于足下。\\
	
\noindent	百尺竿头,更进\xpinyin{一}{yi2}步。
\end{pinyinscope}


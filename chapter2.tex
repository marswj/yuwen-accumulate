\chapter{一年级(下)}
\section{古诗}
\subsection{春晓}
\begin{pinyinscope}
	\noindent \huge 春晓
	
	\noindent {\kaishu \large [唐]孟浩然}  \\
	
	\noindent 春眠不觉晓,\\处处闻啼鸟。\\
	夜来风雨声,\\花落知多少。
\end{pinyinscope}

译文:
春日里贪睡不知不觉天已破晓,搅乱我酣眠的是那啁啾的小鸟。
昨天夜里风声雨声一直不断,那娇美的春花不知被吹落了多少?

\subsection{赠汪伦}
\begin{pinyinscope}
	\noindent \huge 赠汪伦
	
	\noindent {\kaishu \large [唐]李白}  \\
	
	\noindent 李白乘舟将欲行,\\忽闻岸上踏歌声。\\
	桃花潭水深千尺,\\不及汪伦送我情。
\end{pinyinscope}

译文:
我正乘上小船,刚要解缆出发,忽听岸上传来,悠扬踏歌之声。
看那桃花潭水,纵然深有千尺,怎能及汪伦送我之情。

\subsection{静夜思}
\begin{pinyinscope}
	\noindent \huge 静夜思
	
	\noindent {\kaishu \large [唐]李白}  \\
	
	\noindent 床前明月光,\\疑是\xpinyin{地}{di4}上霜。\\
	举头望明月,\\低头思故乡。
\end{pinyinscope}

译文:
明亮的月光洒在床前的窗户纸上,好像地上泛起了一层霜。 我禁不住抬起头来,看那天窗外空中的一轮明月,不由得低头沉思,想起远方的家乡。

\subsection{寻隐者不遇}
\begin{pinyinscope}
	\noindent \huge 寻隐者不遇
	
	\noindent {\kaishu \large [唐]贾岛}  \\
	
	\noindent 松下问童\xpinyin{子}{zi3},\\言师采药去。\\
	只在此山中,\\云深不知处。
\end{pinyinscope}

\noindent 译文:\\
又名:孙革访羊尊师诗。\\苍松下,我询问隐者的童子他的师傅到哪里去了?
他说,师傅已经采药去了。
还指着高山说,就在这座山中,
可是林深云密,我也不知道他到底在哪。

\subsection{池上}
\begin{pinyinscope}
	\noindent \huge 池上
	
	\noindent {\kaishu \large [唐]白居易}  \\
	
	\noindent 小娃撑小艇,\\偷采白莲回。\\
	不解藏踪\xpinyin{迹}{ji4},\\浮萍\xpinyin{一}{yi2}道开。
\end{pinyinscope}

译文:
一个小孩撑着小船,偷偷地采了白莲回来。
他不知道怎么掩藏踪迹,水面的浮萍上留下了一条船儿划过的痕迹。

\subsection{小池}
\begin{pinyinscope}
	\noindent \huge 小池
	
	\noindent {\kaishu \large [宋]杨万里 }  \\
	
	\noindent 泉眼无声惜细流,\\树荫照水爱晴柔。\\
	小荷才露尖尖角,\\早有蜻蜓立上头。
\end{pinyinscope}

译文:
泉眼悄然无声是因舍不得细细的水流,树荫倒映水面是喜爱晴天和风的轻柔。
娇嫩的小荷叶刚从水面露出尖尖的角,早有一只调皮的小蜻蜓立在它的上头。


\subsection{画鸡}
\begin{pinyinscope}
	\noindent \huge 画鸡
	
	\noindent {\kaishu \large [明]唐寅}  \\
	
	\noindent 头上红冠不用裁,\\满身雪白走将来。\\
	平生不敢轻言语,\\ \xpinyin{一}{yi2}叫千门万户开。
\end{pinyinscope}

译文:
头上的红色冠子不用特别剪裁,雄鸡身披雪白的羽毛雄纠纠地走来。
它平生不敢轻易鸣叫,它叫的时候,千家万户的门都打开。

\section{日积月累}
\subsection{积累1}
\begin{pinyinscope}
	\huge
	\noindent 春回大\xpinyin{地}{di4} \quad 万物复苏\\
	柳绿花红 \quad 莺歌燕舞\\
	冰雪融化 \quad 泉水叮咚\\
	百花齐放 \quad 百鸟争鸣
\end{pinyinscope}

\subsection{积累2}
\begin{pinyinscope}
	\huge
	\noindent 小葱拌豆\xpinyin{腐}{fu}---一清(青)二白\\
	竹篮子打水---\xpinyin{一}{yi4}场空\\
	芝\xpinyin{麻}{ma}开花---节节高\\
	十五个吊桶打水---七上八下
\end{pinyinscope}

\subsection{积累3}
\begin{pinyinscope}
	\huge
	\noindent \xpinyin{朝}{zhao1}霞不出门,晚霞行千里。\\
	有雨山戴帽,无雨半山腰。\\
	早\xpinyin{晨}{chen}下雨\xpinyin{当}{dang4}日晴,晚\xpinyin{上}{shang}下雨到天明。\\
	蚂蚁搬家蛇过道,大雨不久要来到。
\end{pinyinscope}

\subsection{积累4}
\begin{pinyinscope}
	\huge
	\noindent 敏而\xpinyin{好}{hao4}学,不耻下问。---{\kaishu \large 《\xpinyin{论}{lun2}语》}\\
	不知则问,不能则学。---{\kaishu \large 《荀\xpinyin{子}{zi3}》}\\
	读书百遍,而义自\xpinyin{见}{xian4}。---{\kaishu \large 董遇}\\
	读万\xpinyin{卷}{juan4}书,行万里路。---{\kaishu \large 董其昌}
\end{pinyinscope}
\chapter{二年级(下)}
\section{古诗}

\subsection{村居}

\noindent 村居

\noindent {\kaishu  [清]高\xpinyin{鼎 }{ding3}}  \\

\noindent 草长\xpinyin{莺}{ying1}飞二月天,\\ \xpinyin{拂}{fu2}\xpinyin{堤}{di1}杨\xpinyin{柳}{liu3}\xpinyin{醉}{zui4}春烟。\\
儿童散学归来早,\\忙趁东风放纸\xpinyin{鸢}{yuan1}。\\


译文:
农历二月,村子前后的青草已经渐渐发芽生长,黄莺飞来飞去。杨柳披着长长的绿枝条,随风摆动,好像在轻轻地抚摸着堤岸。在水泽和草木间蒸发的水汽,如同烟雾般凝集着。杨柳似乎都陶醉在这浓丽的景色中。
村里的孩子们放了学急忙跑回家,趁着东风把风筝放上蓝天。

\subsection{咏柳}

\noindent \xpinyin{咏}{yong3}柳

\noindent {\kaishu  [\xpinyin{唐}{tang2}]\xpinyin{贺}{he4}知\xpinyin{章}{zhang1}  }  \\

\noindent 碧玉\xpinyin{妆}{zhuang1}成一树高,\\万条\xpinyin{垂}{chui2}下绿\xpinyin{丝}{si1}\xpinyin{绦}{tao1}。\\
不知细叶谁\xpinyin{裁}{cai2}出,\\二月春风似\xpinyin{剪}{jian3}刀。

译文:
高高的柳树长满了翠绿的新叶,轻柔的柳枝垂下来,就像万条轻轻飘动的绿色丝带。
这细细的嫩叶是谁的巧手裁剪出来的呢?原来是那二月里温暖的春风,它就像一把灵巧的剪刀。

\subsection{草/赋得古原草送别}

\noindent 草/\xpinyin{赋}{fu4}得古原草送别

\noindent {\kaishu [ \xpinyin{唐}{tang2}]白居\xpinyin{易}{yi4}}  \\

\noindent 离离原上草,一岁一枯\xpinyin{荣}{rong2}。\\
野火烧不尽,春风吹又生。\\
\noindent 远芳侵古道,晴翠接荒城。\\
又送王孙去,萋萋满别情。\\

译文:
长长的原上草是多么茂盛,每年秋冬枯黄春来草色浓。
无情的野火只能烧掉干叶,春风吹来大地又是绿茸茸。\\
野草野花蔓延着淹没古道,艳阳下草地尽头是你征程。我又一次送走知心的好友,茂密的青草代表我的深情。
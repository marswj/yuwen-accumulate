\chapter{二年级(上)}
\section{古诗}
\subsection{梅花}
\begin{pinyinscope}
	\noindent \huge 梅花
	
	\noindent {\kaishu \large [宋]王安石}  \\
	
	\noindent 墙角数枝梅,\\凌寒独自开。\\
	遥知\xpinyin{不}{bu2}是雪,\\为有暗香来。
\end{pinyinscope}

译文:
那墙角的几枝梅花,冒着严寒独自盛开。
为什么远望就知道洁白的梅花不是雪呢?因为梅花隐隐传来阵阵的香气。

\subsection{小儿垂钓}
\begin{pinyinscope}
	\noindent \huge 小儿垂钓
	
	\noindent {\kaishu \large [唐]胡令能}  \\
	
	\noindent 蓬头稚\xpinyin{子}{zi3}学垂纶,\\侧坐莓苔草映身。\\
	路人借问遥招手,\\怕得鱼惊\xpinyin{不}{bu2}\xpinyin{应}{ying4}人。
\end{pinyinscope}

译文:
一个蓬头发蓬乱、面孔青嫩的小孩在河边学钓鱼,侧着身子坐在草丛中,野草掩映了他的身影。
听到有过路的人问路,小孩漠不关心地摆了摆手,生怕惊动了鱼儿,不敢回应过路人。

\subsection{登鹳雀楼}
\begin{pinyinscope}
\noindent \huge 登鹳雀楼

\noindent {\kaishu  \large[唐]王之涣}  \\

\noindent 白日依山\xpinyin{尽}{jin4},\\黄河入海流。\\
欲穷千里目,\\更上\xpinyin{一}{yi4}层楼。
\end{pinyinscope}

译文:
夕阳依傍着西山慢慢地沉没, 滔滔黄河朝着东海汹涌奔流。
若想把千里的风光景物看够, 那就要登上更高的一层城楼。

\subsection{望庐山瀑布}
\begin{pinyinscope}
	\noindent \huge 望庐山瀑布
	
	\noindent {\kaishu \large [唐]李白}  \\
	
	\noindent 日照香炉生紫烟,\\遥看瀑布挂前川。\\
	飞流直下三千尺,\\疑是银河落九天。
\end{pinyinscope}

译文:
香炉峰在阳光的照射下生起紫色烟霞,远远望见瀑布似白色绢绸悬挂在山前。
高崖上飞腾直落的瀑布好像有几千尺,让人恍惚以为银河从天上泻落到人间。

\subsection{江雪}
\begin{pinyinscope}
	\noindent \huge 江雪
	
	\noindent {\kaishu \large [唐]柳宗元}  \\
	
	\noindent 千山鸟飞绝,\\万径人踪灭。\\
	孤舟蓑笠翁,\\独钓寒江雪。
\end{pinyinscope}

译文:
千山万岭不见飞鸟的踪影,千路万径不见行人的足迹。一叶孤舟上,一位身披蓑衣头戴斗笠的渔翁,独自在漫天风雪中垂钓。

\subsection{夜宿山寺}
\begin{pinyinscope}
	\noindent \huge 夜宿山寺
	
	\noindent {\kaishu \large [唐]李白}  \\
	
	\noindent 危楼高百尺,\\手可摘星辰。\\
	不敢高声语,\\恐惊天上人。
\end{pinyinscope}

译文:
山上寺院的高楼真高啊,好像有一百尺的样子,人在楼上好像一伸手就可以摘下天上的星星。
站在这里,我不敢大声说话,唯恐(害怕)惊动天上的神仙。

\subsection{敕勒歌}
\begin{pinyinscope}
\noindent  \huge 敕\xpinyin{勒}{le4}歌

\noindent {\kaishu  \large 北朝民歌 } \\

\noindent 敕\xpinyin{勒}{le4}川,阴山下。\\天\xpinyin{似}{si4}穹庐,\xpinyin{笼}{long3}盖四野。\\
天苍苍,野茫茫。\\风吹草低\xpinyin{见}{xian4}牛羊。
\end{pinyinscope}

译文:
辽阔的敕勒平原,就在千里阴山下,天空仿佛圆顶帐篷,广阔无边,笼罩着四面的原野。
天空蓝蓝的,原野辽阔无边。风儿吹过,牧草低伏,显露出原来隐没于草丛中的众多牛羊。

\section{日积月累}
\subsection{积累1}
\begin{pinyinscope}
	\huge
	\noindent 有山皆图画,无水不文章。\\
	一畦春韭绿,十里稻花香。\\
	忠厚传家久,诗书继世\xpinyin{长}{chang2}。
\end{pinyinscope}

\subsection{积累2}
\begin{pinyinscope}
	\huge
	\noindent 桂林山水甲天下。\\
	上有天堂,下有苏杭。\\
	五岳归来\xpinyin{不}{bu2}看山,黄山归来\xpinyin{不}{bu2}看岳。
\end{pinyinscope}

\subsection{积累3}
\begin{pinyinscope}
	\huge
	\noindent 有志者事竟成。---{\kaishu \large 《后汉书》}\\
	志当存高远。---{\kaishu \large 《诫外生书》}\\
	有志\xpinyin{不}{bu2}在年高。---{\kaishu \large 《传家宝》}\\
\end{pinyinscope}

\subsection{数九歌}
\begin{pinyinscope}
	\huge
	\noindent \xpinyin{数}{shu3}九歌\\
	
	\noindent 一九二九不出手,\\
	三九四九冰上走,\\
	五九六九,沿河看柳,\\
	七九河开,八九雁来,\\
	九九加\xpinyin{一}{yi4}九,耕牛遍\xpinyin{地}{di4}走。
\end{pinyinscope}

\subsection{积累4}
\begin{pinyinscope}
	\huge
	\noindent 狼吞虎咽\\
	龙飞凤舞\\
	鸡鸣狗吠\\
	惊弓之鸟\\
	漏网之鱼\\
	害群之马\\
	胆小如鼠\\
	如虎添翼\\
	如鱼得水
\end{pinyinscope}
\chapter{其他}
\section{古诗积累}
\subsection{村居}

	\noindent 村居
	
	\noindent {\kaishu  [清]高鼎 }  \\
	
	\noindent 草长莺飞二月天,\\拂堤杨柳醉春烟。\\
	儿童散学归来早,\\忙趁东风放纸鸢。\\


译文:
农历二月,村子前后的青草已经渐渐发芽生长,黄莺飞来飞去。杨柳披着长长的绿枝条,随风摆动,好像在轻轻地抚摸着堤岸。在水泽和草木间蒸发的水汽,如同烟雾般凝集着。杨柳似乎都陶醉在这浓丽的景色中。
村里的孩子们放了学急忙跑回家,趁着东风把风筝放上蓝天。

\subsection{悯农(其一)}

\noindent 悯农(其一)

\noindent {\kaishu  [唐]李绅 }  \\

\noindent 春种一粒粟,\\秋收万颗子。\\
四海无闲田,\\农夫犹饿死。

\subsection{赋得古原草送别}

\noindent 赋得古原草送别

\noindent {\kaishu  [唐]白居易}  \\

\noindent 离离原上草,\\一岁一枯荣。\\
野火烧不尽,\\春风吹又生。

译文:
长长的原上草是多么茂盛,每年秋冬枯黄春来草色浓。
无情的野火只能烧掉干叶,春风吹来大地又是绿茸茸。

\subsection{长歌行}

\noindent 长歌行

\noindent {\kaishu  [汉]汉乐府}  \\

\noindent 青青园中葵(\pinyin {kui2}),朝露待日晞(\pinyin {xi1})。\\
阳春布德泽,万物生光辉。\\
常恐秋节至,焜(\pinyin {kun1})黄华叶衰(\pinyin {cui1})。\\
百川东到海,何时复西归?\\
少壮不努力,老大徒伤悲。

译文:
园中的葵菜都郁郁葱葱,晶莹的朝露阳光下飞升。
春天把希望洒满了大地,万物都呈现出一派繁荣。
常恐那肃杀的秋天来到,树叶儿黄落百草也凋零。
百川奔腾着东流到大海,何时才能重新返回西境?
少年人如果不及时努力,到老来只能是悔恨一生。

\subsection{鹿柴}

\noindent 鹿柴

\noindent {\kaishu  [唐]王维 }  \\

\noindent 空山不见人,\\但闻人语响。\\
返景入深林,\\复照青苔上。

译文:
空寂的山中不见一个人,只听到一阵人语声。
太阳的一抹余晖返人深林,又照到林中的青苔上。

\subsection{春夜喜雨}

\noindent 春夜喜雨

\noindent {\kaishu  [唐]杜甫}  \\

\noindent 好雨知时节,当春乃发生。\\
随风潜入夜,润物细无声。\\
野径云俱黑,江船火独明。\\
晓看红湿处,花重锦官城。

译文:
好雨似乎会挑选时辰,降临在万物萌生之春。\\
伴随和风,悄悄进入夜幕。细细密密,滋润大地万物。\\
浓浓乌云,笼罩田野小路,点点灯火,闪烁江上渔船。\\
明早再看带露的鲜花,成都满城必将繁花盛开。

\subsection{游子吟}

\noindent 游子吟

\noindent {\kaishu  [唐]孟郊}  \\

\noindent 慈母手中线,游子身上衣。\\
临行密密缝,意恐迟迟归。\\
谁言寸草心,报得三春晖。

译文:
慈母用手中的针线,为远行的儿子赶制身上的衣衫。\\
临行前一针针密密地缝缀,怕的是儿子回来得晚衣服破损。\\
有谁敢说,子女像小草那样微弱的孝心,能够报答得了像春晖普泽的慈母恩情呢?

\subsection{忆江南}

\noindent 忆江南

\noindent {\kaishu  [唐]白居易}  \\

\noindent 江南好,风景旧曾谙。\\
日出江花红胜火,春来江水绿如蓝。能不忆江南?

译文:
江南的风景多么美好,如画的风景久已熟悉。
春天到来时,太阳从江面升起,把江边的鲜花照得比火红,碧绿的江水绿得胜过蓝草。怎能叫人不怀念江南?

\subsection{绝句}

\noindent 绝句

\noindent {\kaishu  [唐]杜甫 }  \\

\noindent 迟日江山丽,\\春风花草香。\\
泥融飞燕子,\\沙暖睡鸳鸯。

译文:
江山沐浴着春光,多么秀丽,春风送来花草的芳香。
燕子衔着湿泥忙筑巢,暖和的沙子上睡着成双成对的鸳鸯。

\subsection{咏柳}

\noindent 咏柳

\noindent {\kaishu  [唐]贺知章  }  \\

\noindent 碧玉妆成一树高,\\万条垂下绿丝绦。\\
不知细叶谁裁出,\\二月春风似剪刀。

译文:
高高的柳树长满了翠绿的新叶,轻柔的柳枝垂下来,就像万条轻轻飘动的绿色丝带。
这细细的嫩叶是谁的巧手裁剪出来的呢?原来是那二月里温暖的春风,它就像一把灵巧的剪刀。

\subsection{清明}

\noindent 清明

\noindent {\kaishu  [唐]杜牧}  \\

\noindent 清明时节雨纷纷,\\路上行人欲断魂。\\
借问酒家何处有?\\牧童遥指杏花村。

译文:
清明节这天细雨纷纷,路上远行的人好像断魂一样迷乱凄凉。
问一声牧童哪里才有酒家,他指了指远处的杏花村。

\subsection{明日歌}

\noindent 明日歌

\noindent {\kaishu  [明]钱福}  \\

\noindent 明日复明日,明日何其多。\\
我生待明日,万事成蹉跎。\\
世人若被明日累,春去秋来老将至。\\
朝看水东流,暮看日西坠。\\
百年明日能几何?请君听我明日歌。\\
明日复明日,明日何其多!\\
日日待明日,万事成蹉跎。\\
世人皆被明日累,明日无穷老将至。\\
晨昏滚滚水东流,今古悠悠日西坠。\\
百年明日能几何?请君听我明日歌。

译文:
明天又一个明天,明天何等的多。 我的一生都在等待明日,什么事情都没有进展。 世人和我一样辛苦地被明天所累,一年年过去马上就会老。 早晨看河水向东流逝,傍晚看太阳向西坠落才是真生活。 百年来的明日能有多少呢?请诸位听听我的《明日歌》。

\subsection{游子吟}

\noindent 游子吟

\noindent {\kaishu  [唐]孟郊}  \\

\noindent 慈母手中线,游子身上衣。\\
临行密密缝,意恐迟迟归。\\
谁言寸草心,报得三春晖。

译文:
慈母用手中的针线,为远行的儿子赶制身上的衣衫。
临行前一针针密密地缝缀,怕的是儿子回来得晚衣服破损。
有谁敢说,子女像小草那样微弱的孝心,能够报答得了像春晖普泽的慈母恩情呢?

\subsection{相思}

\noindent 相思

\noindent {\kaishu  [唐]王维}  \\

\noindent 红豆生南国,\\春来发几枝。\\
愿君多采撷(\pinyin {xie2}),\\此物最相思。

译文:
鲜红浑圆的红豆,生长在阳光明媚的南方,春暖花开的季节,不知又生出多少?
希望思念的人儿多多采集,小小红豆引人相思。

\subsection{出塞}

\noindent 出塞

\noindent {\kaishu  [唐]王昌龄}  \\

\noindent 秦时明月汉时关,\\万里长征人未还。\\
但使龙城飞将在,\\不教胡马度阴山。

译文:
依旧是秦汉时期的明月和边关,守边御敌鏖战万里征人未回还。
倘若龙城的飞将李广如今还在,绝不许匈奴南下牧马度过阴山。

\subsection{七步诗}

\noindent 七步诗

\noindent {\kaishu  [三国·魏]曹植}  \\

\noindent 煮豆持作羹,漉菽以为汁。\\
萁在釜下燃,豆在釜中泣。\\
本自同根生,相煎何太急?

译文:
锅里煮着豆子,是想把豆子的残渣过滤出去,留下豆汁来作羹。
豆秸在锅底下燃烧,豆子在锅里面哭泣。
豆子和豆秸本来是同一条根上生长出来的,豆秸怎能这样急迫地煎熬豆子呢?
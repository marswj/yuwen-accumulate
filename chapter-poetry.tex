\chapter{其他}
\section{古诗积累}

\subsection{悯农(其一)}

\noindent 悯农(其一)

\noindent {\kaishu  [唐]李绅 }  \\

\noindent 春种一粒粟,\\秋收万颗子。\\
四海无闲田,\\农夫犹饿死。



\subsection{长歌行}

\noindent 长歌行

\noindent {\kaishu  [汉]汉乐府}  \\

\noindent 青青园中葵(\pinyin {kui2}),朝露待日晞(\pinyin {xi1})。\\
阳春布德泽,万物生光辉。\\
常恐秋节至,焜(\pinyin {kun1})黄华叶衰(\pinyin {cui1})。\\
百川东到海,何时复西归?\\
少壮不努力,老大徒伤悲。

译文:
园中的葵菜都郁郁葱葱,晶莹的朝露阳光下飞升。
春天把希望洒满了大地,万物都呈现出一派繁荣。
常恐那肃杀的秋天来到,树叶儿黄落百草也凋零。
百川奔腾着东流到大海,何时才能重新返回西境?
少年人如果不及时努力,到老来只能是悔恨一生。

\subsection{鹿柴}

\noindent 鹿\xpinyin{柴}{zhai4}

\noindent {\kaishu  [唐]王维 }  \\

\noindent 空山不见人,\\但闻人语响。\\
返景入深林,\\复照青苔上。

译文:
空寂的山中不见一个人,只听到一阵人语声。
太阳的一抹余晖返人深林,又照到林中的青苔上。

\subsection{春夜喜雨}

\noindent 春夜喜雨

\noindent {\kaishu  [唐]杜甫}  \\

\noindent 好雨知时节,当春乃发生。\\
随风潜入夜,润物细无声。\\
野径云俱黑,江船火独明。\\
晓看红湿处,花重锦官城。

译文:
好雨似乎会挑选时辰,降临在万物萌生之春。\\
伴随和风,悄悄进入夜幕。细细密密,滋润大地万物。\\
浓浓乌云,笼罩田野小路,点点灯火,闪烁江上渔船。\\
明早再看带露的鲜花,成都满城必将繁花盛开。

\subsection{游子吟}

\noindent 游子吟

\noindent {\kaishu  [唐]孟郊}  \\

\noindent 慈母手中线,游子身上衣。\\
临行密密缝,意恐迟迟归。\\
谁言寸草心,报得三春晖。

译文:
慈母用手中的针线,为远行的儿子赶制身上的衣衫。\\
临行前一针针密密地缝缀,怕的是儿子回来得晚衣服破损。\\
有谁敢说,子女像小草那样微弱的孝心,能够报答得了像春晖普泽的慈母恩情呢?

\subsection{忆江南}

\noindent 忆江南

\noindent {\kaishu  [唐]白居易}  \\

\noindent 江南好,风景旧曾谙。\\
日出江花红胜火,春来江水绿如蓝。能不忆江南?

译文:
江南的风景多么美好,如画的风景久已熟悉。
春天到来时,太阳从江面升起,把江边的鲜花照得比火红,碧绿的江水绿得胜过蓝草。怎能叫人不怀念江南?

\subsection{绝句}

\noindent 绝句

\noindent {\kaishu  [唐]杜甫 }  \\

\noindent 迟日江山丽,\\春风花草香。\\
泥融飞燕子,\\沙暖睡鸳鸯。

译文:
江山沐浴着春光,多么秀丽,春风送来花草的芳香。
燕子衔着湿泥忙筑巢,暖和的沙子上睡着成双成对的鸳鸯。

\subsection{绝句}

\noindent 绝句

\noindent {\kaishu  [唐]杜甫 }  \\

\noindent 两个黄鹂鸣翠柳,一行白鹭上青天。\\
窗含西岭千秋雪,门泊东吴万里船。\\

译文:
两只黄鹂在翠绿的柳树间婉转地歌唱,一队整齐的白鹭直冲向蔚蓝的天空。我坐在窗前,可以望见西岭上堆积着终年不化的积雪,门前停泊着自万里外的东吴远行而来的船只。



\subsection{清明}

\noindent 清明

\noindent {\kaishu  [唐]杜牧}  \\

\noindent 清明时节雨纷纷,\\路上行人欲断魂。\\
借问酒家何处有?\\牧童遥指杏花村。

译文:
清明节这天细雨纷纷,路上远行的人好像断魂一样迷乱凄凉。
问一声牧童哪里才有酒家,他指了指远处的杏花村。

\subsection{明日歌}

\noindent 明日歌

\noindent {\kaishu  [明]钱福}  \\

\noindent 明日复明日,明日何其多。\\
我生待明日,万事成蹉跎。\\
世人若被明日累,春去秋来老将至。\\
朝看水东流,暮看日西坠。\\
百年明日能几何?请君听我明日歌。\\
明日复明日,明日何其多!\\
日日待明日,万事成蹉跎。\\
世人皆被明日累,明日无穷老将至。\\
晨昏滚滚水东流,今古悠悠日西坠。\\
百年明日能几何?请君听我明日歌。

译文:
明天又一个明天,明天何等的多。 我的一生都在等待明日,什么事情都没有进展。 世人和我一样辛苦地被明天所累,一年年过去马上就会老。 早晨看河水向东流逝,傍晚看太阳向西坠落才是真生活。 百年来的明日能有多少呢?请诸位听听我的《明日歌》。

\subsection{游子吟}

\noindent 游子吟

\noindent {\kaishu  [唐]孟郊}  \\

\noindent 慈母手中线,游子身上衣。\\
临行密密缝,意恐迟迟归。\\
谁言寸草心,报得三春晖。

译文:
慈母用手中的针线,为远行的儿子赶制身上的衣衫。
临行前一针针密密地缝缀,怕的是儿子回来得晚衣服破损。
有谁敢说,子女像小草那样微弱的孝心,能够报答得了像春晖普泽的慈母恩情呢?

\subsection{相思}

\noindent 相思

\noindent {\kaishu  [唐]王维}  \\

\noindent 红豆生南国,\\春来发几枝。\\
愿君多采撷(\pinyin {xie2}),\\此物最相思。

译文:
鲜红浑圆的红豆,生长在阳光明媚的南方,春暖花开的季节,不知又生出多少?
希望思念的人儿多多采集,小小红豆引人相思。

\subsection{出塞}

\noindent 出塞

\noindent {\kaishu  [唐]王昌龄}  \\

\noindent 秦时明月汉时关,\\万里长征人未还。\\
但使龙城飞将在,\\不教胡马度阴山。

译文:
依旧是秦汉时期的明月和边关,守边御敌鏖战万里征人未回还。
倘若龙城的飞将李广如今还在,绝不许匈奴南下牧马度过阴山。

\subsection{七步诗}

\noindent 七步诗

\noindent {\kaishu  [三国·魏]曹植}  \\

\noindent 煮豆持作羹,漉菽以为汁。\\
萁在釜下燃,豆在釜中泣。\\
本自同根生,相煎何太急?

译文:
锅里煮着豆子,是想把豆子的残渣过滤出去,留下豆汁来作羹。
豆秸在锅底下燃烧,豆子在锅里面哭泣。
豆子和豆秸本来是同一条根上生长出来的,豆秸怎能这样急迫地煎熬豆子呢?

\subsection{一字诗}

\noindent 一字诗

\noindent {\kaishu  [清]陈沆(\pinyin {hang4})}  \\

\noindent 一帆一桨一渔舟,\\
一个渔翁一钓钩。\\
一俯一仰一场笑,\\
一江明月一江秋。

\subsection{晓窗}

\noindent 晓窗

\noindent {\kaishu  [清]魏源}  \\

\noindent 少闻鸡声眠,\\老听鸡声起。\\
千古万代人,\\消磨数声里。

译文:
由少到老,世上千千万万代人,他们的岁月与生命,都无一例外地消磨在报晓的鸡鸣中,无志者消沉,蹉跎岁月;有志者奋发,建功立业。人生短促,时不我待。

\subsection{山行}

\noindent 山行

\noindent {\kaishu  [唐] 杜牧}  \\

\noindent 远上寒山石径斜,\\白云生处有人家。\\
停车坐爱枫林晚,\\霜叶红于二月花。

译文:
沿着弯弯曲曲的小路上山,在那白云深处,居然还有人家。停下车来,是因为喜爱这深秋枫林晚景。枫叶秋霜染过,艳比二月春花。

\subsection{黄鹤楼送孟浩然之广陵}

\noindent 黄鹤楼送孟浩然之广陵

\noindent {\kaishu  [唐] 李白}  \\

\noindent 故人西辞黄鹤楼,\\烟花三月下扬州。\\
孤帆远影碧空尽,\\唯见长江天际流。

译文:
老朋友向我频频挥手,告别了黄鹤楼,在这柳絮如烟、繁花似锦的阳春三月去扬州远游。友人的孤船帆影渐渐地远去,消失在碧空的尽头,只看见一线长江,向邈远的天际奔流。

\subsection{枫桥夜泊 / 夜泊枫江}

\noindent 枫桥夜泊 / 夜泊枫江

\noindent {\kaishu  [唐] 张继}  \\

\noindent 月落乌啼霜满天,\\江枫渔火对愁眠。\\
姑苏城外寒山寺,\\夜半钟声到客船。

译文:
月亮已落下乌鸦啼叫寒气满天,对着江边枫树和渔火忧愁而眠。姑苏城外那寂寞清静寒山古寺,半夜里敲钟的声音传到了客船。

\subsection{古朗月行}

\noindent 古朗月行

\noindent {\kaishu  [唐] 李白}  \\

\noindent 小时不识月,呼作白玉盘。\\
又疑瑶台镜,飞在青云端。\\
仙人垂两足,桂树何团团。\\
白兔捣药成,问言与谁餐。\\
蟾蜍蚀圆影,大明夜已残。\\
羿昔落九乌,天人清且安。\\
阴精此沦惑,去去不足观。\\
忧来其如何,凄怆摧心肝。

译文:
小时候不认识月亮,把它称为白玉盘。又怀疑是瑶台仙镜,飞在夜空青云之上。月中的仙人是垂着双脚吗?月中的桂树为什么长得圆圆的?白兔捣成的仙药,到底是给谁吃的呢?蟾蜍把圆月啃食得残缺不全,皎洁的月儿因此晦暗不明。后羿射下了九个太阳,天上人间免却灾难清明安宁。月亮已经沦没而迷惑不清,没有什么可看的不如远远走开吧。心怀忧虑啊又何忍一走了之,凄惨悲伤让我肝肠寸断。

\subsection{望洞庭}

\noindent 望洞庭

\noindent {\kaishu  [唐] 刘禹锡 }  \\

\noindent 湖光秋月两相和,\\潭面无风镜未磨。\\
遥望洞庭山水翠,\\白银盘里一青螺。

译文:
风静浪息,月光和水色交融在一起,湖面就像不用磨拭的铜镜,平滑光亮。遥望洞庭,山青水绿,林木葱茏的洞庭山耸立在泛着白光的洞庭湖里,就像白银盘里的一只青螺。

\subsection{望天门山}

\noindent 望天门山

\noindent {\kaishu  [唐] 李白}  \\

\noindent 天门中断楚江开,\\碧水东流至此回。\\
两岸青山相对出,\\孤帆一片日边来。

译文:
长江犹如巨斧劈开天门雄峰,碧绿江水东流到此没有回旋。两岸青山对峙美景难分高下,遇见一叶孤舟悠悠来自天边。

\subsection{墨梅}

\noindent 墨梅

\noindent {\kaishu  [元] 王冕}  \\

\noindent 吾家洗砚池头树,\\朵朵花开淡墨痕。\\
不要人夸颜色好,\\只留清气满乾坤。

译文:
我家洗砚池边有一棵梅树,朵朵开放的梅花都显出淡淡的墨痕。不需要别人夸它的颜色好看,只需要梅花的清香之气弥漫在天地之间。

\subsection{竹石}

\noindent 竹石

\noindent {\kaishu  [清] 郑\xpinyin{燮}{xie4}(郑板桥)}  \\

\noindent 咬定青山不放松,\\立根原在破岩中。\\
千磨万击还坚劲,\\任尔东西南北风。

译文:
紧紧咬定青山不放松,原本深深扎根石缝中。千磨万击身骨仍坚劲,任凭你刮东西南北风。

\subsection{早发白帝城 / 白帝下江陵}

\noindent 早发白帝城 / 白帝下江陵

\noindent {\kaishu  [唐] 李白}  \\

\noindent 朝辞白帝彩云间,\\千里江陵一日还。\\
两岸猿声啼不住,\\轻舟已过万重山。

译文:
清晨,朝霞满天,我就要踏上归程。从江上往高处看,可以看见白帝城彩云缭绕,如在云间,景色绚丽!千里之遥的江陵,一天之间就已经到达。两岸猿猴的啼声不断,回荡不绝。猿猴的啼声还回荡在耳边时,轻快的小船已驶过连绵不绝的万重山峦。

\subsection{回乡偶书}

\noindent 回乡偶书

\noindent {\kaishu  [唐]贺知章} \\

\noindent 少小离家老大回,\\乡音无改\xpinyin{鬓}{bin4}毛\xpinyin{衰}{shuai1}。\\
儿童相见不相识,\\笑问客从何处来。

译文:
我在年少时离开家乡,到了迟暮之年才回来。我的乡音虽未改变,但鬓角的毛发却已经疏落。儿童们看见我,没有一个认识的。他们笑着询问:这客人是从哪里来的呀?